\section{Conclusion}
\label{sec:conclusion}
In conclusion, it can be observed that a trained neural network can reach an 
accuracy of $99\%$ by setting a sufficient number of epochs. In addition, a 
good training and validation dataset is necessary, but not sufficient to 
guarantee the result, given that it is supervised learning as seen in the 
section \ref{subsec:supervised-learnig}, it is necessary to spend time to 
prepare meaningful images.
On the other hand, it is not possible to train a neural network on a home 
computer much less on a laptop as a result it is necessary to use a HPC with 
nodes with performing GPUs, and at present they are guaranteed by Nvidia 
hardware and CUDA libraries.
At present there is no single encoding to save the neural network model, this 
necessarily implies that the network must be converted from time to time in 
the most appropriate format depending on the chosen framework.
This fact implies that some operations or information are not correctly 
represented or in the worst case unsupported.
Thanks to the API made available by Intel, it is possible to quickly 
prototyping and deepening neural networks for the Movidius neural stick on the 
computer used for development and also on SoC devices such as Raspberry.
However, a still unripe software is highlighted, in fact it is necessary:
\begin{itemize}
\item a specific version of Linux Ubuntu, in fact only version 16.04 is 
supported, the higher version are not supported
\item The Keras framework is not officially support, in fact some model 
conversions are needed.
\item Some operations performed on a neural network inherent in the framework 
are not supported.
\end{itemize}
Operation on Raspberry is difficult because the \emph{Raspbian} operating system
is a derivative of Debian Linux and does not officially support TensorFlow.
Although the demos released by Intel on the Coffee framework work, custom 
models based on other frameworks, as in our case TensorFlow, may present 
malfunctions due to dependencies and porting of libraries.
