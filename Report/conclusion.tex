\section{Conclusion}
\label{sec:conclusion}
In conclusion, it can be observed that a trained neural network can reach an 
accuracy of $99\%$ by setting a sufficient number of epochs. In addition, a 
good training and validation dataset is necessary, but not sufficient to 
guarantee the result, given that it is a supervised learning as seen in the 
section \ref{subsec:supervised-learnig}.
Hence, it is necessary to prepare meaningful images.\\
On the other hand, it is not possible to train a neural network using a home 
computer neither a laptop; a HPC is needed and all its computing nodes have to be
equipped with GPUs. 
Nowadays, GPUs are guaranteed by Nvidia hardware and CUDA libraries.
At present there is no single encoding to save the neural network model, this 
necessarily implies that the network must be converted from time to time in 
the most appropriate format, depending on the chosen framework.
This fact entail that some operations or information are not correctly 
represented or, in the worst case, unsupported.
Thanks to the API made available by Intel, it is possible to  
prototyping and deepening neural networks for the Movidius neural stick on the 
 development computer and on SoC devices, such as Raspberry.
However, it is still a  not-whole-developed software; hence, to run, it requires:
\begin{itemize}
\item a specific version of Linux Ubuntu - only the version 16.04 is 
supported, while the higher versions are not supported;
\item the convertion of a model, because the Keras framework is not officially supported;
\item the modification or the elimination of some operations available in the 
framework because they are not supported.
\end{itemize}
Operation on Raspberry is difficult because the \emph{Raspbian} operating system
is a derivative of Debian Linux and it does not officially support TensorFlow.
Although the demos, released by Intel and based on the Caffe framework, run 
normally, the custom models may present malfunctions due to dependencies and 
porting libraries, because they are developed through other frameworks (as in 
our case TensorFlow).
